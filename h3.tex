\documentclass[10pt]{book}
\usepackage{graphicx}
\usepackage{subfig} % make it possible to include more than one captioned figure/table in a single float
\usepackage[utf8]{inputenc}
\usepackage{hyperref}
\usepackage[intlimits]{amsmath}
\usepackage{amssymb}
\usepackage{tkz-euclide}
\usepackage{tikz}
\setlength{\oddsidemargin}{15.5pt} 
\setlength{\evensidemargin}{15.5pt}
\pretolerance=2000
\tolerance=3000
\renewcommand{\figurename}{Figura}
\renewcommand{\chaptername}{Cap\'{i}tulo}
\renewcommand{\contentsname}{\'{I}ndice}
\renewcommand{\tablename}{Tabla}
\renewcommand{\bibname}{Bibliograf\'{i}a}
\renewcommand{\appendixname}{Ap\'endices}


\usepackage{geometry}
 \geometry{
 a4paper,
 left=15mm,
 right=10mm,
 top=20mm,
 bottom=20mm,
 }

\begin{document}
\paragraph 2

Notacion para la configuración de los electrones de un átomo:
\begin{description}
\item $^{2S+1}L_{J}$, pero en lugar del valor de L se usan letras:
\item $S \equiv L =0 $
\item $P \equiv L =1 $
\item $D \equiv L =2 $
\item $F \equiv L =3 $
\item donde los números cuanticos: S representa  el spin, 
L el momento angular orbital y J el momento angular total (spin + orbital) considerando todos los electrones del átomo
\item el valor del factor Landé para cada nivel de la transición:

$g = 1 + \frac{J(J+1) + S(S+1) - L(L+1)}{2J(J+1)}$ si $J \neq 0$ y $g=0$ si $J=0$

\item el valor del factor Landé efectivo de la transición:

$\bar{g} = \frac{1}{2}(g_1 + g_2) + \frac{1}{4}(g_1 - g_2)(J_1(J_1+1) - J_2(J_2 + 1))$

donde los valores $_1$ son del nivel antes de la transición y los valores $_2$ después


\end{description}

\begin{enumerate}
\item $5D_2 - 7D_3$

$5D_2 \implies S_1 = 2, L_1 = 2, J_1 = 2 \implies g_1 = 1 + \frac{6+6-6}{12} = \frac{3}{2}$

$7D_3 \implies S_2 = 3, L_2 = 2, J_2 = 3 \implies g_2 = 1 + \frac{12+12-6}{24} = \frac{7}{4}$

$\bar{g} = 2$

\item $5D_0 - 7D_1$

$5D_0 \implies S_1 = 2, L_1 = 2, J_1 = 0 \implies g_1 = 0$

$7D_1 \implies S_2 = 3, L_2 = 2, J_2 = 1 \implies g_2 = 1 + \frac{2+12-6}{4} = 3$

$\bar{g} = 3$

\item $5F_1 - 5D_0$

$5F_1 \implies S_1 = 2, L_1 = 3, J_1 = 1 \implies g_1 = 1 + \frac{2+6-12}{4} = 0$

$5D_0 \implies S_2 = 2, L_2 = 2, J_2 = 0 \implies g_2 = 0$

$\bar{g} = 0$

\item $5P_2 - 5D_2$

$5P_2 \implies S_1 = 2, L_1 = 1, J_1 = 2 \implies g_1 =  \frac{11}{6}$

$5D_2 \implies S_2 = 2, L_2 = 2, J_2 = 2 \implies g_2 = \frac{3}{2}$

$\bar{g} = \frac{5}{3}$

\item $5P_1 - 5D_0$

$5P_1 \implies S_1 = 2, L_1 = 1, J_1 = 1 \implies g_1 = \frac{5}{2}$

$5D_0 \implies S_2 = 2, L_2 = 2, J_2 = 0 \implies g_2 = 0$

$\bar{g} = \frac{5}{2}$

\end{enumerate}

 
https://github.com/beevageeva/fsol/







\end{document}
