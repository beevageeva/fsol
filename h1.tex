\documentclass[10pt]{book}
\usepackage{graphicx}
\usepackage{subfig} % make it possible to include more than one captioned figure/table in a single float
\usepackage[utf8]{inputenc}
\usepackage{hyperref}
\usepackage[intlimits]{amsmath}
\usepackage{amssymb}
\usepackage{tkz-euclide}
\usepackage{tikz}
\setlength{\oddsidemargin}{15.5pt} 
\setlength{\evensidemargin}{15.5pt}
\pretolerance=2000
\tolerance=3000
\renewcommand{\figurename}{Figura}
\renewcommand{\chaptername}{Cap\'{i}tulo}
\renewcommand{\contentsname}{\'{I}ndice}
\renewcommand{\tablename}{Tabla}
\renewcommand{\bibname}{Bibliograf\'{i}a}
\renewcommand{\appendixname}{Ap\'endices}


\usepackage{geometry}
 \geometry{
 a4paper,
 left=15mm,
 right=10mm,
 top=20mm,
 bottom=20mm,
 }

\begin{document}
\paragraph 1 
El sentido de las fuerzas gravitatoria($\vec{F_g}$), centrífuga($\vec{F_c}$), debida al gradiente de presión de radiación($\vec{F_{Prad}}$) y del gas($\vec{F_P}$)que accionan en el punto Q situado al radio r (en coordenadas esféricas, suponiendo simetría esférica):
\begin{tikzpicture}[scale=.8]

% definitions
\tkzDefPoint(0,0){O}
\tkzDefPoint(3.5,0){Q}
\tkzDefPoint(2.5,0){Fg}
\tkzDefPoint(4.5,0){Fc}
\tkzDefPoint(6.0,0){Fprad}
\tkzDefPoint(7.5,0){Fp}

\draw [->, shorten >= -5cm, shorten <=-5cm] (O)--(Q);
\draw [->] (Q)--(Fc);
\draw [->] (Fc)--(Fprad);
\draw [->] (Fprad)--(Fp);
\draw [<-] (Fg)--(Q);
\node at (10,0.5){r} ; 
\node at (4.5,0.5){$\vec{F_{c}}$} ; 
\node at (6.0,0.5){$\vec{F_{Prad}}$} ; 
\node at (7.5,0.5){$\vec{F_{P}}$} ; 
\node at (2.5,0.5){$\vec{F_{g}}$} ; 
\tkzDrawCircle(O,Q) 
\tkzDrawPoints(O,Q)

% labels
\tkzLabelPoints(Q,O)
\end{tikzpicture}

Por la simetría esférica las variables $\rho$, P, T, .. no dependen de los ángulos $\theta$ y $\phi$ 

$\nabla_r f  = \frac{df}{dr} = \frac{\partial f}{\partial r}$

$\nabla_\theta f  = \nabla_\phi f = 0$

(solo la componente en la dirección $\vec{e_r}$ del gradiente es diferente de 0)

Notamos $F_g$, $F_c$, $F_{Prad}$, $F_P$ los módulos de estas fuerzas por unidad de masa (acceleraciones)

La ecuación de equilibrio hidrostático:

$F_g = F_P + F_c + F_{Prad}$

donde 

$F_g = \frac {G M(r)}{r^2}$


$F_P = -\frac {1}{\rho}\frac {\partial P}{\partial r}$


$F_{Prad} = -\frac {1}{\rho}\frac {\partial Prad}{\partial r}$

$Prad = \frac{4 \sigma T^4}{3 c}$

$\implies$

$F_{Prad} = -\frac {16 \sigma T^3}{3 c \rho}\frac {\partial T}{\partial r}$


Los gradientes de temperatura y presión son negativos lo que justifica el signo -


$F_c = \frac {v_r^2}{r}$ donde $v_r$ es el módulo de la velocidad de rotación (la dirección es tangente a la esfera en Q)

$v_r = \frac{2 \pi r}{A} $ donde A es el periodo de rotación de 27 días(A = 27 * 24 * 3600 s) $\implies$

$F_c = \frac {4 \pi^2 r}{A^2}$

Si despreciamos $F_c$ y  $F_{Prad}$ la ecuación del equilibrio hidrostático:

$ \frac{d P}{d r} = -\rho \frac{G M(r)}{r^2}$

Usamos las notaciones y valores:
\begin{description}
\item $M_S = 1.9891 \cdot 10^{30} kg$ para la masa del sol
\item $R_S = 0.7 \cdot 10^9 m$ para el radio del sol
\item $\sigma = 5.67 \cdot 10^{-8} \frac{W}{m^2 K^4}$ para la constante de Stefan-Boltzmann
\item $G = 6.674 \cdot 10^{-11} \frac{N m^2}{kg^2}$ para la constante de gravitación universal
\item $c = 3 \cdot 10^{8} \frac{m}{s}$ para la velocidad de la luz

\end{description}

Notamos $q = \frac{r}{R_S}$ y $m_f = \frac{M(r)}{M_S}$ 
y podemos considerar las variables P, $\rho$, T, $m_f$ de la tabla del modelo estándar como funciones de $q$

Hay que estimar los valores de $F_c$, $F_g$ y $F_{Prad}$ para  $q \in \{0.1, 0.3, 1\}$

$F_c = \frac {4 \pi^2 R_S}{A^2} q$
 

$F_g = \frac {G M_S}{R_S^2} \frac{m_f}{q^2} $

$\frac {\partial T}{\partial r} = \frac {\partial T}{\partial q}   \frac {\partial q}{\partial r} =  \frac {1}{R_S}  \frac {\partial T}{\partial q} \implies $

$F_{Prad} = -\frac {16 \sigma}{3 c R_S} \frac {T^3}{\rho} \frac {\partial T}{\partial q}$

En la tabla tenemos n + 1 valores discretos de q: $q_0=0, q_1, .. q_n = 1$
y de las funciones f(q), con f que quede ser T, $\rho$, $m_f$,.. 
 
Para calcular los valores de estas funciones y de las derivadas $\forall q \in [0,1]$:

\begin{description}
\item caso1:

$\exists i \in \{0,..n\} | q = q_i$

\begin{description}
\item $f(q) = f(q_i)$
\begin{description}
\item $i<n \implies \frac {\partial f}{\partial q} = \frac {f(q_{i+1}) - f(q_i)}{q_{i+1} - q_i}$
\item $i=n \implies \frac {\partial f}{\partial q} = \frac {f(q_{n}) - f(q_{n-1})}{q_{n} - q_{n-1}}$
\end{description}
\end{description}

\item caso2:

$\exists i \in \{1,..n\} | q \in (q_{i-1},q_i)$

\begin{description}

\item $f(q) = f(q_{i-1}) + \frac{(q  - q_{i-1})(f(q_i) - f(q_{i-1}))}{q_i - q_{i-1}}$

(interpolación lineal)

\item $ \frac {\partial f}{\partial q} = \frac {f(q_{i}) - f(q_{i-1})}{q_{i} - q_{i-1}}$
\end{description}

\end{description}
En el caso del problema: $q \in \{0, 0.3, 1\}$ 
\small

\begin{verbatim}

q=1.0e-01
mf=7.8184e-02
rho=8.7594e+04 kg/m3
T=1.3073e+07 K
dT/dq=-3.7875e+07 K
dT/dr=-5.4107e-02 K/m
Fg=2.1182e+03 m/s2
Fc=5.0781e-04 m/s2
Fc/Fg=2.3974e-07
Fprad=1.3910e+00 m/s2
Fprad/Fg=6.5670e-04

q=3.0e-01
mf=6.0629e-01
rho=1.2288e+04 kg/m3
T=6.8316e+06 K
dT/dq=-2.1250e+07 K
dT/dr=-3.0357e-02 K/m
Fg=1.8251e+03 m/s2
Fc=1.5234e-03 m/s2
Fc/Fg=8.3472e-07
Fprad=7.9396e-01 m/s2
Fprad/Fg=4.3503e-04

q=1.0e+00
mf=1.0000e+00
rho=2.5120e-04 kg/m3
T=5.7780e+03 K
dT/dq=-2.4630e+07 K
dT/dr=-3.5186e-02 K/m
Fg=2.7092e+02 m/s2
Fc=5.0781e-03 m/s2
Fc/Fg=1.8744e-05
Fprad=2.7236e-02 m/s2
Fprad/Fg=1.0053e-04


\end{verbatim}

\normalsize

En los 3 casos $\frac{F_c}{F_g} < \frac{F_Prad}{F_g}  \approx 0$

\paragraph 2

$ \epsilon = 2.54 \cdot 10^4 \rho X^2 T_9^{-\frac{2}{3}} e^{-\frac{3.37}{T_9^\frac{1}{3}}} \frac{erg}{g \cdot s} $


S.I.: 

$ 1erg = 10^{-7} J $,

$ 1g = 10^{-3} kg $


$ \epsilon = 2.54 \cdot  \rho X^2 T_9^{-\frac{2}{3}} e^{-\frac{3.37}{T_9^\frac{1}{3}}} \frac{J}{kg \cdot s} $

$ E_{nuc} = 2.54 \cdot  \rho X^2 T_9^{-\frac{2}{3}} e^{-\frac{3.37}{T_9^\frac{1}{3}}} \cdot M_S \cdot mf_{nuc}$   W (1W = 1J/s)

Calculamos para el núcleo: $\frac{r}{R_S} \le 2.0130e-01$ y $\frac{r}{R_S} \le 2.3760e-01$ 

\begin{verbatim}

rfNuc= Rnuc/RS = 2.0130e-01
mfNuc= Mnuc/MS = 3.3880e-01
T=1.3114e+07 K
T9=1.3114e-02 K
rho=9.7816e+04 kg/m3
X=5.0200e-01
Enuc = 4.717933e+29 W


rfNuc= Rnuc/RS = 2.3760e-01
mfNuc= Mnuc/MS = 4.4520e-01
T=1.2579e+07 K
T9=1.2579e-02 K
rho=8.9590e+04 kg/m3
X=5.2333e-01
Enuc = 5.195219e+29 W

\end{verbatim}


\paragraph 3

$ \mu  = \frac{1}{\sum_j \frac{X_j}{\mu_j}} $ donde $X_j$ es la abundancia de cada elemento

$ \mu_j  = \frac{A_j}{N_j} $ donde $N_j$ es el número de partículas / núcleo ($N_j$ = 1 núcleo + núm. electrones libres) y 
$A_j$ es la masa atómica del elemento($A_j$  = núm protones + num. neutrones del núcleo)

Para los elementos totalmente ionizados(todos los electrones libres) $\mu_j = \frac{A_j}{1+Z_j}$ donde $Z_j$ = núm protones del núcleo(núm atómico) y para los metales $\mu_j \approx 2$

Hay que considerar para cado estado de ionización un término en la suma con abundancia $p_{j_i} \cdot X_j $ donde $p_{j_i}$ es la proporción de estos iones de todos los átomos de este elemento ($A_j$ va a ser el mismo, pero $N_j$ va a ser diferente para cada estado de ionización)

Consideramos $A_{met}$ = 17

\begin{description}
\item H, He estado neutro

$ \mu_H  = \frac{1}{1} = 1$ ($N_H$ = 1 núcleo, $A_H$ = 1)

$ \mu_{He}  = \frac{4}{1} = 4$ ($N_{He}$ = 1 núcleo, $A_{He}$ = 4)

$ \mu = \frac{1}{X + \frac{Y}{4}}$
 
\item H, He totalmente ionizados

$ \mu_{H^+}  = \frac{1}{2} $ ($N_H$ = 1 núcleo + 1$e^-$ = 2, $A_H$ = 1)

$ \mu_{He^{+2}}  = \frac{4}{3} $ ($N_{He}$ = 1 núcleo + 2 $e^-$ = 3, $A_{He}$ = 4)

$ \mu = \frac{1}{2X + \frac{3Y}{4}}$

\item H, He y metales en estado neutro

$ \mu_H  = \frac{1}{1} = 1$ ($N_H$ = 1 núcleo, $A_H$ = 1)

$ \mu_{He}  = \frac{4}{1} = 4$ ($N_{He}$ = 1 núcleo, $A_{He}$ = 4)

$ \mu_{met}  = \frac{17}{1} = 17$ ($N_{met}$ = 1 núcleo, $A_{met}$ = 17)

$ \mu = \frac{1}{X + \frac{Y}{4} + \frac{Z}{17}}$

\item H, He totalmente ionizados y una fracción 50\% de metales  totalmente ionizados

$ \mu_{H^+}  = \frac{1}{2} $ ($N_H$ = 1 núcleo + 1$e^-$ = 2, $A_H$ = 1)

$ \mu_{He^{+2}}  = \frac{4}{3} $ ($N_{He}$ = 1 núcleo + 2 $e^-$ = 3, $A_{He}$ = 4)

$ \mu_{met}  = \frac{17}{1} = 17 $ ($N_{met}$ = 1 núcleo , $A_{met}$ = 17)

$ \mu_{met^{+Z_{met}}}  \approx 2  $ 

$ \mu = \frac{1}{2 X + \frac{3 Y}{4} + 0.5 \cdot \frac{Z}{17} + 0.5 \cdot \frac{Z}{2} } = \frac{1}{2 X + \frac{3 Y}{4} + \frac{19 Z}{68} }$

\item X = 0.7346, Y = 0.2485, Z = 0.0169

Para el caso H,He y metales en estado neutro:  $ \mu = 1.2536$

Para el caso H,He totalmente ionizados y metales ionizados totalmente 50\%:  $\mu = 0.6023$

$\mu$ está creciendo con el radio porque a radio menor donde la temperatura es mas alta los elementos están mas ionizados y los electrones libres se suman a $N_j$ y $ \mu_j $ es mas pequeño

Mirando en la tabla del modelo estándar intentamos a encontrar valores de $\mu \in [0.611, 1.247]$ (X = 0.735) y los valores de los 2 casos de arriba no cuadran


$ \mu_H  = \frac{1}{1} = 1$ ($N_H$ = 1 núcleo, $A_H$ = 1)

$ \mu_{H^{+1}}  = \frac{1}{2} $ ($N_H$ = 1 núcleo + 1 $e^-$ = 2, $A_H$ = 1)

$ \mu_{He}  = \frac{4}{1} = 4$ ($N_{He}$ = 1 núcleo, $A_{He}$ = 4)

$ \mu_{He^{+1}}  = \frac{4}{2} = 2$ ($N_{He}$ = 1 núcleo + 1 $e^-$ = 2, $A_{He}$ = 4)

$ \mu_{met}  = \frac{17}{1} = 17$ ($N_{met}$ = 1 núcleo, $A_{met}$ = 17)

$ \mu_{met^{+1}}  = \frac{17}{2}$ ($N_{met}$ = 1 núcleo + 1 $e^{-}$ = 2, $A_{met}$ = 17)

Considerando  metales una vez ionizados en proporción p y H y He solo en forma de átomos neutros:

$ \mu = \frac{1}{(X + \frac{Y}{4}  (1-p)\frac{Z}{17} + 2 p \frac{Z}{17}}$

$ X + \frac{Y}{4} +\frac{Z}{17}+  p \frac{Z}{17} = \frac{1}{\mu} $

$ p = \frac{17}{Z} (\frac{1}{\mu} - X - \frac{Y}{4} - \frac{Z}{17}) $

$p(\mu = 1.247) = 4.2303 > 1$ lo que implica que hay que considerar mas estados de ionización

Intentamos encontrar una solución considerando H, He y metales una vez ionizados en la misma proporción p

$ \mu = \frac{1}{(1-p)X + 2p X+ (1-p)\frac{Y}{4} + p \frac{Y}{2}+ (1-p)\frac{Z}{17} + 2 p \frac{Z}{17}}$

$ X + \frac{Y}{4} +\frac{Z}{17}+  p (X + \frac{Y}{4} + \frac{Z}{17}) = \frac{1}{\mu} $

$ p = \frac{\frac{1}{\mu} - X - \frac{Y}{4} - \frac{Z}{17}}{X + \frac{Y}{4} + \frac{Z}{17}} $

$p(\mu = 0.611) = 1.05168 > 1, p(\mu = 1.247) = 5.2719e-03, \mu(p = 1) = 6.2679e-01 $ lo que 
implica que con un grado pequeño de ionización ($p \ge 2719e-3$) de todos los elementos (solo consideramos una vez ionizados) 
obtenemeos valores de $\mu$ coherentes con la tabla desde  $\mu = 1.247$  en la superficie  hasta llegar  a $\mu = 0.6267$ y $p=1$, 
pero si queremos simular valores mas pequeños de $\mu$ (a radio menor, temperatura mayor)
 (y $\mu \ge 0.611$ por el valor de X) hay que considerar los elementos mas veces ionizados 


\item Tenemos de las ecuaciones  de Saha p la proporción de hidrógeno ionizado y  $q_1$, $q_2$ las proporciones de helio ionizado (una vez y 2 veces ionizado)

$ \mu_H  = \frac{1}{1} = 1$ ($N_H$ = 1 núcleo, $A_H$ = 1)

$ \mu_{H^{+1}}  = \frac{1}{2} $ ($N_H$ = 1 núcleo + 1 $e^-$ = 2, $A_H$ = 1)

$ \mu_{He}  = \frac{4}{1} = 4$ ($N_{He}$ = 1 núcleo, $A_{He}$ = 4)

$ \mu_{He^{+1}}  = \frac{4}{2} = 2$ ($N_{He}$ = 1 núcleo + 1 $e^-$ = 2, $A_{He}$ = 4)

$ \mu_{He^{+2}}  = \frac{4}{3} $ ($N_{He}$ = 1 núcleo + 2 $e^-$ = 3, $A_{He}$ = 4)
 
$ \mu = \frac{1}{(1-p)X + 2pX + (1-q_1-q_2)  \frac{Y}{4} + q_1 \frac{Y}{2} + q_2 \frac{3Y}{4}}$

$ \mu = \frac{1}{X(1 + p) + Y (\frac{1}{4}  + \frac{q_1}{4} + \frac{q_2}{2})}$

\end{description}


 
https://github.com/beevageeva/fsol/







\end{document}
