\documentclass[10pt]{book}
\usepackage{graphicx}
\usepackage{subfig} % make it possible to include more than one captioned figure/table in a single float
\usepackage[utf8]{inputenc}
\usepackage{hyperref}
\usepackage[intlimits]{amsmath}
\usepackage{amssymb}
\usepackage{tkz-euclide}
\usepackage{tikz}
\setlength{\oddsidemargin}{15.5pt} 
\setlength{\evensidemargin}{15.5pt}
\pretolerance=2000
\tolerance=3000
\renewcommand{\figurename}{Figura}
\renewcommand{\chaptername}{Cap\'{i}tulo}
\renewcommand{\contentsname}{\'{I}ndice}
\renewcommand{\tablename}{Tabla}
\renewcommand{\bibname}{Bibliograf\'{i}a}
\renewcommand{\appendixname}{Ap\'endices}


\usepackage{geometry}
 \geometry{
 a4paper,
 left=15mm,
 right=10mm,
 top=20mm,
 bottom=20mm,
 }

\begin{document}
\paragraph 1
$
\begin{pmatrix}
I\\Q\\U\\V
\end{pmatrix}
= 
\alpha
\begin{pmatrix}
I_{m}\\Q_{m}\\U_{m}\\V_{m}
\end{pmatrix}
+ (1-\alpha)
\begin{pmatrix}
I_{nm}\\Q_{nm}\\U_{nm}\\V_{nm}
\end{pmatrix}
$

$Q_{nm}=U_{nm}=V_{nm} = 0$ (luz no polarizada $\implies$ los componentes $E_1$ y $E_2$ incoherentes 
$\implies <E_1 E_1^{*}> = <E_2 E_2^{*}>$ y  $<E_1 E_2^{*}> = <E_2 E_1^{*}>=0$; luego mirar las definiciones de los parámetros Stokes)

$\implies
\begin{pmatrix}
I\\Q\\U\\V
\end{pmatrix}
= 
\begin{pmatrix}
\alpha I_{m} + (1-\alpha) I_{nm}\\\alpha Q_{m}\\\alpha U_{m}\\\alpha V_{m}
\end{pmatrix}
$

Para la parte magnética en el campo débil: (del pdf escaneado)

$V_m = -\Delta\lambda_B cos\theta \frac{dI_m}{d\lambda}$

($\Delta \lambda_B$ (de la hoja) = $\bar{g} \lambda_B$ (en el pdf pag 103) y ver  la equación de V en la pag. 110 - weak field solution)

Suponemos $I_m = I_{nm} \implies I = I_{m}$

$V = \alpha V_m$

$\implies V = -\alpha \Delta\lambda_B cos\theta \frac{dI}{d\lambda}$


\paragraph 2

Notacion para la configuración de los electrones de un átomo:
\begin{description}
\item $^{2S+1}L_{J}$
\item pero en lugar del valor de L se usan letras:
\item $S \equiv L =0 $
\item $P \equiv L =1 $
\item $D \equiv L =2 $
\item $F \equiv L =3 $
\item donde los números cuanticos: S representa  el spin, 
L el momento angular orbital y J el momento angular total (spin y momento angular orbital) 
considerando todos los electrones del átomo
\item el valor del factor Landé para cada nivel de la transición:

$g = 1 + \frac{J(J+1) + S(S+1) - L(L+1)}{2J(J+1)}$ si $J \neq 0$ y $g=0$ si $J=0$

\item el valor del factor Landé efectivo de la transición:

$\bar{g} = \frac{1}{2}(g_1 + g_2) + \frac{1}{4}(g_1 - g_2)(J_1(J_1+1) - J_2(J_2 + 1))$

donde los valores $_1$ son del nivel antes de la transición y los valores $_2$ después


\end{description}

\begin{enumerate}
\item $5D_2 - 7D_3$ (Fe I, $\lambda = 5247.1$ A)

$5D_2 \implies S_1 = 2, L_1 = 2, J_1 = 2 \implies g_1 = 1 + \frac{6+6-6}{12} = \frac{3}{2}$

$7D_3 \implies S_2 = 3, L_2 = 2, J_2 = 3 \implies g_2 = 1 + \frac{12+12-6}{24} = \frac{7}{4}$

$\bar{g} = 2$

\item $5D_0 - 7D_1$ (Fe I, $\lambda = 5250.2$ A)

$5D_0 \implies S_1 = 2, L_1 = 2, J_1 = 0 \implies g_1 = 0$

$7D_1 \implies S_2 = 3, L_2 = 2, J_2 = 1 \implies g_2 = 1 + \frac{2+12-6}{4} = 3$

$\bar{g} = 3$

\item $5F_1 - 5D_0$ (Fe I, $\lambda = 5576.1$ A)

$5F_1 \implies S_1 = 2, L_1 = 3, J_1 = 1 \implies g_1 = 1 + \frac{2+6-12}{4} = 0$

$5D_0 \implies S_2 = 2, L_2 = 2, J_2 = 0 \implies g_2 = 0$

$\bar{g} = 0$ (magnetic insensitive line)

\item $5P_2 - 5D_2$ (Fe I, $\lambda = 6301.5$ A)

$5P_2 \implies S_1 = 2, L_1 = 1, J_1 = 2 \implies g_1 =  \frac{11}{6}$

$5D_2 \implies S_2 = 2, L_2 = 2, J_2 = 2 \implies g_2 = \frac{3}{2}$

$\bar{g} = \frac{5}{3}$

\item $5P_1 - 5D_0$ (Fe I, $\lambda = 6302.5$ A)

$5P_1 \implies S_1 = 2, L_1 = 1, J_1 = 1 \implies g_1 = \frac{5}{2}$

$5D_0 \implies S_2 = 2, L_2 = 2, J_2 = 0 \implies g_2 = 0$

$\bar{g} = \frac{5}{2}$

\end{enumerate}


\paragraph 3

$\Delta \lambda_B = k \lambda_0^2 g B $

donde $k = 4.67 \cdot 10^{-13} A^{-1} G^{-1} = 4.67 \cdot 10^{-3} m^{-1} G^{-1}$

y g es el factor Landé efectivo de la transición ($\bar{g}$)

$\Delta \lambda_D = \Delta \lambda_B \iff B = \frac{v}{c k \lambda_0 g} $


donde v es la velocidad total (con los componentes de la velocidad térmica y microturbulencia) que aparece en
la fórmula del ensanchamiento Doppler

\begin{verbatim}
Line: Halpha , Lambda: 6562.8 A, g = 1.0
T = 15000.0 K, v = 1.1212e+04 m/s, dlD = 0.2453 A
B = 200.0 G, dlB = 0.0040 A
B = 1000.0 G, dlB = 0.0201 A
B = 3000.0 G, dlB = 0.0603 A
dlD = dlB <=> B = 12194.6568 G

Line: FeI , Lambda: 6302.5 A, g = 2.5
T = 5000.0 K, v = 1.3200e+03 m/s, dlD = 0.0277 A
B = 200.0 G, dlB = 0.0093 A
B = 1000.0 G, dlB = 0.0464 A
B = 3000.0 G, dlB = 0.1391 A
dlD = dlB <=> B = 597.9685 G

Line: FeI , Lambda: 15648.0 A, g = 3.0
T = 6000.0 K, v = 1.3751e+03 m/s, dlD = 0.0717 A
B = 200.0 G, dlB = 0.0686 A
B = 1000.0 G, dlB = 0.3430 A
B = 3000.0 G, dlB = 1.0291 A
dlD = dlB <=> B = 209.0781 G
\end{verbatim}

 
\paragraph 4

$\Delta \lambda_B = k \lambda_0^2 g B $

$\Delta \lambda_D =  \lambda_0 \frac{v}{c} $

$S_m = \frac{\Delta \lambda_B}{\Delta \lambda_D} = C \lambda_0 g$

donde $C = \frac{k B c}{v}$ no depende de $\lambda_0$ o g 

$\implies S_m \propto \lambda_0 g$

https://github.com/beevageeva/fsol/







\end{document}
