\documentclass[12pt]{book}
\usepackage{graphicx}
\usepackage{subfig} % make it possible to include more than one captioned figure/table in a single float
\usepackage[utf8]{inputenc}
\usepackage{hyperref}
\usepackage[intlimits]{amsmath}
\usepackage{amssymb}
\usepackage{float}
\setlength{\oddsidemargin}{15.5pt} 
\setlength{\evensidemargin}{15.5pt}
\pretolerance=2000
\tolerance=3000
\renewcommand{\figurename}{Figura}
\renewcommand{\chaptername}{Cap\'{i}tulo}
\renewcommand{\contentsname}{\'{I}ndice}
\renewcommand{\tablename}{Tabla}
\renewcommand{\bibname}{Bibliograf\'{i}a}
\renewcommand{\appendixname}{Ap\'endices}


\usepackage{geometry}
 \geometry{
 a4paper,
 left=5mm,
 right=5mm,
 top=5mm,
 bottom=5mm,
 }

\begin{document}

\paragraph{H1 p9}
\begin{equation}
\int_0^{r_0}{4\pi r^2 n_e(r) dr } = Z
\end{equation}


$n_e(r) = \frac{8 \pi}{3 h^3} [2 m_e (e_F + e V(r))]^{\frac{3}{2}}$  (1.29 apuntes)

$x = \frac{r}{\mu a_0} \implies r = x a_0 (\frac{9 \pi^2}{128 Z})^{\frac{1}{3}}$

$\Phi(x) = \frac{e_F + e V(r)}{\frac{Z e^2}{4 \pi \epsilon_0 r}} \implies $

$e_F + e V(r) = \Phi(x) \frac{Z e^2}{4 \pi \epsilon_0 r} \implies $

$n_e(r) = \frac{8 \pi}{3 h^3} (2 m_e \Phi(x) \frac{Z e^2}{4 \pi \epsilon_0 r} )^{\frac{3}{2}}$ 

reemplazando en (1):

$\int_0^{r_0}{4\pi r^2  \frac{8 \pi}{3 h^3} (\Phi(x) \frac{m_e Z e^2}{2 \pi \epsilon_0 r} )^{\frac{3}{2}}dr } = Z \implies$

$\frac{32 \pi^2}{3 h^3} (\frac{m_e Z e^2}{2 \pi \epsilon_0})^{\frac{3}{2}} \int_0^{r_0}{r^{\frac{1}{2}} \Phi(x)^{\frac{3}{2}} dr } = Z$

Cambio de variable r por x 

$dr = dx a_0 (\frac{9 \pi^2}{128 Z})^{\frac{1}{3}}$


$\frac{32 \pi^2}{3 h^3} (\frac{m_e Z e^2}{2 \pi \epsilon_0})^{\frac{3}{2}} a_0^{\frac{3}{2}}  (\frac{9 \pi^2}{128 Z})^{\frac{1}{2}}\int_0^{x_0}{x^{\frac{1}{2}} \Phi(x)^{\frac{3}{2}} dx } = Z $

$\Phi(x)^{\frac{3}{2}} = x^{\frac{1}{2}} \frac{d^2\Phi}{dx^2} \implies$

$\frac{32 \pi^2}{3 h^3} (\frac{m_e Z e^2}{2 \pi \epsilon_0})^{\frac{3}{2}} a_0^{\frac{3}{2}}  (\frac{9 \pi^2}{128 Z})^{\frac{1}{2}}\int_0^{x_0}{x \frac{d^2\Phi}{dx^2} dx } = Z \implies$


$\frac{32 \pi^2}{3 h^3} (\frac{m_e  e^2}{2 \pi \epsilon_0})^{\frac{3}{2}} a_0^{\frac{3}{2}}  (\frac{9 \pi^2}{128 })^{\frac{1}{2}}\int_0^{x_0}{x \frac{d^2\Phi}{dx^2} dx } = 1 \implies$

Notamos C = $(\frac{32 \pi^2}{3 h^3} (\frac{m_e  e^2}{2 \pi \epsilon_0})^{\frac{3}{2}} a_0^{\frac{3}{2}}  (\frac{9 \pi^2}{128 })^{\frac{1}{2}})^{-1} $

$\int_0^{x_0}{x \frac{d^2\Phi}{dx^2} dx } = C $

Integrando por partes:

$\int_0^{x_0}{x \frac{d^2\Phi}{dx^2} dx } = (x \Phi'(x))|_0^{x_0} - \int_0^{x_0}{\Phi'(x) dx } = x_0 \Phi'(x_0) - \Phi(x_0) + \Phi(0) \implies$

$x_0 \Phi'(x_0) - \Phi(x_0) = C - 1$ ($\impliedby \Phi(0)=1$)

$C = 1$ ($\impliedby a_0 = \frac{4 \pi \epsilon_0 \hbar^2}{m_e e^2}$ )

 
$\implies  \Phi'(x_0) = \frac{ \Phi(x_0)}{x_0}$

\paragraph{H1 p11}


all ionized $\implies \frac{1}{\mu} = 2 X + \frac{3}{4} Y + \frac{1}{2} Z$

$\mu  = 1.3793$ g/mol

eq 1.40, $M = M_{\odot} = 1.99 \cdot 10^{33}g$ $\implies$

$C = 6.65 \cdot 10^4 \frac{\mu}{Z(1+X)} = 91.7241 \cdot 10^4$ erg $s^{-1} K^{\frac{-7}{2}}$ 

$T_c = (\frac{L}{C})^{\frac{2}{7}}$

$L = 0.03 L_{\odot} = 0.117 \cdot 10^{33} $ erg/s $\implies$

$T_c = 2.0697 \cdot 10^6$ K

$T_s = (\frac{C}{4 \pi R^2 \sigma})^{\frac{1}{4}} T_c^{\frac{7}{8}}$

$\sigma = 5.67 \cdot 10^{-5}$ erg $cm^{-2} K^{-4} s^{-1}$

$R = R_{\odot} = 6.96 \cdot 10^{10}$ cm

$\implies T_s = 2412.9238$ K

$\kappa_0 = 4.34 \cdot 10^{22} Z (1+X) cm^2 g^{-1} = 4.34 \cdot 10^{21} cm^2 g^{-1}$

$\rho_s = (\frac{4}{17} \frac{64 \sigma \pi}{3} \frac{GM}{L} \frac{\mu m_H }{\kappa_0 k_B} )^{\frac{1}{2}} \cdot T_c^{\frac{13}{4}}$

$k_B = 1.38 \cdot 10^{-16} erg K^{-1}, G = 6.6725 \cdot 10^{-8} cm^3 g^{-1} s^{-2}, m_H = 1.6733 \cdot 10^{-24} g$

$\implies \rho_s = 21.0285 g cm^{-3}$

Notamos $\rho_c$ = densidad en la base de la capa (la temperatura mas abajo de este punto - hasta el centro es constante = $T_c$, pero la densidad no)

$T_s \rho_s^{-\frac{2}{3}} = T_c \rho_c^{-\frac{2}{3}}$

$\implies \rho_c = \rho_s (\frac{T_s}{T_c})^{-\frac{3}{2}} $

$\implies \rho_c = 528269.89561g cm^{-3} $

\paragraph{H1 p12}

$M = 0.6 M_{\odot} = 1.194 \cdot 10^{33}$ g

$t = 1.4 \cdot 10^{10}$ años = $4.415 \cdot 10^{17}$ s

$T_0 = 3 \cdot 10^7$ K

carbono $\implies$ A = 12, $\mu$ = 12 g/mol, X = Y = 0, Z = 1

$C = 6.65 \cdot 10^4 \frac{M}{M_{\odot}} \frac{\mu}{Z(1+X)} erg s^{-1}K^{-\frac{7}{2}} = 47.88 \cdot 10^4  erg s^{-1}K^{-\frac{7}{2}}$ 

$\tau_0 = \frac{3}{2} \frac{M k_B}{A m_H C T_0^{\frac{5}{2}}}$ s

$\tau_0 = 0.05215 \cdot 10^{17}$ s

$\frac{L}{L_0} = (1+\frac{5}{2} \frac{t}{\tau_0} )^{-\frac{7}{5}} = 0.00055115$

\paragraph{H2 p4}

caso no relativista (baja densidad: $\rho << 6  \cdot 10^{15} $ g/$cm^3$)

$\gamma = 5/3$, $K = \frac{3^{\frac{2}{3}} \pi^{\frac{4}{3}} \hbar^2 }{5 m_n^{\frac{8}{3}}} = 5.38752 \cdot 10^9$

$\gamma = 1 + \frac{1}{n} \implies n = \frac{3}{2}$

en la ecuación Lane Emden n = 1.5 igual que en el caso de las enanas blancas  de baja densidad $\implies $ tiene la misma resolución:
$\xi_1= 3.65375$ y $|\theta'(\xi_1)|= 0.203302$  

polítropos apuntes eq 1.18, 1.19 de cuales resulta la relación masa radio:

$M = 4 \pi R^{\frac{n-3}{n-1}} (\frac{(n+1)K}{4\pi G})^{\frac{n}{n+1}} \xi_1^{\frac{n+1}{n-1}} |\theta'(\xi_1)| $

$\implies M R^3 = 0.88 \cdot 10^{13} g cm^3$

Si la masa no dependiera de $\rho_c$ como en el caso de las enanas blancas relativistas(n=3) la masa de la relación 1.19 sería la masa límite.

Pero en este caso M depende de $\rho_c$ (polítropos relación 1.19) $\implies$ no hay masa límite

(este caso de estrellas de neutrones es identico en este problema al de las enanas blancas de baja masa (no relativistas), pero con K diferente ( en las ecuaciones de los polítropos) )

\paragraph{H3 p2}

partícula de masa = 1 parte del  reposo de una distancia R 

$\implies E = c^2 \sqrt{1-\frac{r_s}{R}}$ 

eq 3.6 apuntes: $ c^2(1-\frac{r_s}{r}) \frac{dt}{d\tau} = E \implies \frac{d\tau}{dt} = (1-\frac{r_s}{r})^{\frac{1}{2}}  $

apuntes (parte de una distancia R): $\tau(r) = \frac{1}{c} (\frac{R^3}{r_s})^{\frac{1}{2}} [(\frac{r}{R} -\frac{r^2}{R^2})^{\frac{1}{2}} + arccos(\sqrt{\frac{r}{R}}) ] $

$r = R \frac{1+cos \eta}{2} \implies \tau(\eta) = \frac{1}{c} (\frac{R^3}{r_s})^{\frac{1}{2}} [(\frac{1+cos \eta}{2} -(\frac{1+cos \eta}{2}) ^2)^{\frac{1}{2}} + arccos(\sqrt{\frac{1+cos \eta}{2}}) ] $

$\frac{d\tau}{d\eta} =  \frac{1}{c} (\frac{R^3}{r_s})^{\frac{1}{2}} ( \frac{\frac{1}{2} \sin (\eta) (\cos (\eta)+1)-\frac{\sin (\eta)}{2}}{2 \sqrt{\frac{1}{2} (\cos (\eta)+1)-\frac{1}{4} (\cos (\eta)+1)^2}} + \frac{\sin (\eta)}{2 \sqrt{2} \sqrt{\frac{1}{2} (-\cos (\eta)-1)+1} \sqrt{\cos (\eta)+1}} )$





$\frac{d\tau}{d\eta} \frac{d\eta}{dt} = (1-\frac{r_s}{r})^{\frac{1}{2}} \implies \frac{dt}{d\eta} = \frac{1}{c} (\frac{R^3}{r_s})^{\frac{1}{2}} 
\frac{\frac{\sin (\eta)}{2 \sqrt{1-\frac{1}{4} (\cos (\eta)+1)^2}}+\frac{\frac{1}{2} \sin (\eta) (\cos (\eta)+1)-\frac{\sin (\eta)}{2}}{2 \sqrt{\frac{1}{2} (\cos (\eta)+1)-\frac{1}{4} (\cos (\eta)+1)^2}}}{\sqrt{1-\frac{2 r_s}{R (\cos (\eta)+1)}}}
$

$\implies t(\eta) = \frac{1}{c} (\frac{R^3}{r_s})^{\frac{1}{2}} 
\frac{\sqrt{\sin ^2(\eta)} \sec ^2\left(\frac{\eta}{2}\right) \left(\sqrt{R} (R \cos (\eta)+R-2 r_s)+\sqrt{2} (R+r_s) \csc \left(\frac{\eta}{2}\right) \sqrt{R \cos (\eta)+R-2 r_s} \tan ^{-1}\left(\frac{\sqrt{2} \sqrt{R} \sin \left(\frac{\eta}{2}\right)}{\sqrt{R \cos (\eta)+R-2 r_s}}\right)\right)}{4 R^{3/2} \sqrt{1-\frac{2 r_s}{R \cos (\eta)+R}}} $

con $csc(x) = \frac{1}{sin(x)}$ y $sec(x) = \frac{1}{cos(x)}$

(cálculos hechos con mathematica)

singularidad en $r=r_s$ ($R(cos \eta +1) = 2 r_s$ y el denominador se hace 0)

\paragraph{H3 p4}

partícula con masa m = 1 parte del reposo desde el infinito $\implies E = c^2$ 

$\frac{dr}{d\tau} = -c (\frac{r_s}{r})^{\frac{1}{2}} \implies$

$r^{\frac{1}{2}} dr = -c r_s^{\frac{1}{2}} d\tau \implies$

$\tau(r) = C -\frac{2}{3}c^{-1}r_s^{-\frac{1}{2}} r^{\frac{3}{2}} $

$\tau(R) = 0 \implies C =  \frac{2}{3}c^{-1}r_s^{-\frac{1}{2}} R^{\frac{3}{2}} $

$\tau(r) = \frac{2}{3}c^{-1}r_s^{-\frac{1}{2}} R^{\frac{3}{2}}  -\frac{2}{3}c^{-1}r_s^{-\frac{1}{2}} r^{\frac{3}{2}} $

$\tau(r_s) = \frac{2}{3}c^{-1}r_s^{-\frac{1}{2}} R^{\frac{3}{2}}  -\frac{2}{3}c^{-1}r_s^{-\frac{1}{2}} r_s^{\frac{3}{2}} = \frac{2}{3}c^{-1}r_s^{-\frac{1}{2}} R^{\frac{3}{2}}  -\frac{2}{3}c^{-1}r_s$  (tiempo propio(medido con el reloj que viaja con la partícula) para llegar a $r_s$)


$\tau(0) = \frac{2}{3}c^{-1}r_s^{-\frac{1}{2}} R^{\frac{3}{2}}$ (tiempo propio para llegar al centro)

velocidad propia $v(r) = \frac{dr}{d\tau} = -c (\frac{r_s}{r})^{\frac{1}{2}} $ (negativa porque va hacía el centro (de coordenadas y del agujero negro)) $\implies v(r_s) = -c$


\paragraph{H3 p9}

la energía total de un agujero negro = energía debia a la masa + energía de rotación:

$M c^2 = M_{irr} c^2 + E_{rot}$

$M = M(A,J) $ la energía(total) es función de A y J y se puede escribir como

$M = \sqrt{\frac{A c^4}{16 \pi G^2} + \frac{4 \pi J^2}{c^2 A} }$

donde notamos  $M_{irr} = \sqrt{\frac{c^4 A}{16 \pi G^2}}$

al final del capítulo de Agujeros negros de los apuntes para determinar la energía de rotación máxima:

se deduce que $M_{irr} $ mínima (cuando la energía de rotación es máxima) es $\frac{1}{\sqrt{2}} M$

la condición que el área resultante no puede dsiminuir $A \ge 2 A_1$

conservación de energía (los 2 agujeros que se funden en uno):

$2 M_1 c^2 = M c^2$

$\implies$ Los límites de $M_{irr}$ del agujero resultante dependen de $E_{rot}$ (la energía total $M = 2M_1$ fijo)

$M_{irr}$ es mínimo cuando $E_{rot}$ es máximo (ver la explicación de los apuntes)
$\implies \frac{M_{irr}}{M} \ge \frac{1}{\sqrt{2}} \implies \frac{M_{irr}}{2M_1} \ge \frac{1}{\sqrt{2}} $

$M_{irr}$ es máximo cuando $E_{rot}$ es mínimo = 0 $\implies max(M_{irr}) = M = 2 M_1 \implies \frac{M_{irr}}{2 M_1}\le 1$

b) $M_{irr} = 2 M_1$ (tiene el valor máximo) cuando no hay rotación: $a= 0 \implies J = 0 (J \le J_{max} \forall J_{max} \ge 0)$. 

en general $J_{max}$ es determinado por la condición $a_{max} = \frac{r_s}{2}$

los agujeros iniciales no tienen rotación $\implies M_1 = M_{irr1} \implies M_1 =  \sqrt{\frac{c^4 A_1}{16 \pi G^2}}$

$J_{max} = M c a_{max} = 2 M_1 c \frac{r_s}{2}  = \frac{r_s c^3}{4G} \sqrt{\frac{A_1}{\pi}}$

El área se calcula en los 2 casos $M_{irr} = \sqrt{\frac{c^4 A}{16 \pi G^2}} \implies A = \frac{16 \pi G^2 M_{irr}^2}{c^4}$

$M_{irr} = 2 M_1 \implies A = \frac{64 \pi G^2 M_1}{c^4}$

c)agujero en rotación máxima: $M_{irr} = M_1 \sqrt{2} \implies A = \frac{32 \pi G^2 M_1}{c^4}$

la energía de la masa en reposo se transforma en energía de rotación

\end{document}
