\documentclass[12pt]{book}
\usepackage{graphicx}
\usepackage{subfig} % make it possible to include more than one captioned figure/table in a single float
\usepackage[utf8]{inputenc}
\usepackage{hyperref}
\usepackage[intlimits]{amsmath}
\usepackage{amssymb}
\usepackage{float}
\setlength{\oddsidemargin}{15.5pt} 
\setlength{\evensidemargin}{15.5pt}
\pretolerance=2000
\tolerance=3000
\renewcommand{\figurename}{Figura}
\renewcommand{\chaptername}{Cap\'{i}tulo}
\renewcommand{\contentsname}{\'{I}ndice}
\renewcommand{\tablename}{Tabla}
\renewcommand{\bibname}{Bibliograf\'{i}a}
\renewcommand{\appendixname}{Ap\'endices}


\usepackage{geometry}
 \geometry{
 a4paper,
 left=5mm,
 right=5mm,
 top=5mm,
 bottom=5mm,
 }

\begin{document}

\paragraph{H1 p9}
\begin{equation}
\int_0^{r_0}{4\pi r^2 n_e(r) dr } = Z
\end{equation}


$n_e(r) = \frac{8 \pi}{3 h^3} [2 m_e (e_F + e V(r))]^{\frac{3}{2}}$  (1.29 apuntes)

$x = \frac{r}{\mu a_0} \implies r = x a_0 (\frac{9 \pi^2}{128 Z})^{\frac{1}{3}}$

$\Phi(x) = \frac{e_F + e V(r)}{\frac{Z e^2}{4 \pi \epsilon_0 r}} \implies $

$e_F + e V(r) = \Phi(x) \frac{Z e^2}{4 \pi \epsilon_0 r} \implies $

$n_e(r) = \frac{8 \pi}{3 h^3} (2 m_e \Phi(x) \frac{Z e^2}{4 \pi \epsilon_0 r} )^{\frac{3}{2}}$ 

reemplazando en (1):

$\int_0^{r_0}{4\pi r^2  \frac{8 \pi}{3 h^3} (\Phi(x) \frac{m_e Z e^2}{2 \pi \epsilon_0 r} )^{\frac{3}{2}}dr } = Z \implies$

$\frac{32 \pi^2}{3 h^3} (\frac{m_e Z e^2}{2 \pi \epsilon_0})^{\frac{3}{2}} \int_0^{r_0}{r^{\frac{1}{2}} \Phi(x)^{\frac{3}{2}} dr } = Z$

Cambio de variable r por x ($dr = dx a_0 (\frac{9 \pi^2}{128 Z})^{\frac{1}{3}}$)


$\frac{32 \pi^2}{3 h^3} (\frac{m_e Z e^2}{2 \pi \epsilon_0})^{\frac{3}{2}} a_0^{\frac{3}{2}}  (\frac{9 \pi^2}{128 Z})^{\frac{1}{2}}\int_0^{x_0}{x^{\frac{1}{2}} \Phi(x)^{\frac{3}{2}} dx } = Z $

$\Phi(x)^{\frac{3}{2}} = x^{\frac{1}{2}} \frac{d^2\Phi}{dx^2} \implies$

$\frac{32 \pi^2}{3 h^3} (\frac{m_e Z e^2}{2 \pi \epsilon_0})^{\frac{3}{2}} a_0^{\frac{3}{2}}  (\frac{9 \pi^2}{128 Z})^{\frac{1}{2}}\int_0^{x_0}{x \frac{d^2\Phi}{dx^2} dx } = Z \implies$


$\frac{32 \pi^2}{3 h^3} (\frac{m_e  e^2}{2 \pi \epsilon_0})^{\frac{3}{2}} a_0^{\frac{3}{2}}  (\frac{9 \pi^2}{128 })^{\frac{1}{2}}\int_0^{x_0}{x \frac{d^2\Phi}{dx^2} dx } = 1 \implies$

Notamos C = $(\frac{32 \pi^2}{3 h^3} (\frac{m_e  e^2}{2 \pi \epsilon_0})^{\frac{3}{2}} a_0^{\frac{3}{2}}  (\frac{9 \pi^2}{128 })^{\frac{1}{2}})^{-1} $

$\int_0^{x_0}{x \frac{d^2\Phi}{dx^2} dx } = C $

Integrando por partes:

$\int_0^{x_0}{x \frac{d^2\Phi}{dx^2} dx } = (x \Phi'(x))|_0^{x_0} - \int_0^{x_0}{\Phi'(x) dx } = x_0 \Phi'(x_0) - \Phi(x_0) + \Phi(0) \implies$

$x_0 \Phi'(x_0) - \Phi(x_0) = C - 1$

$C = 1$ 

\paragraph{H1 p11}


all ionized $\implies \frac{1}{\mu} = 2 X + \frac{3}{4} Y + \frac{1}{2} Z$

$\frac{1}{\mu}  = 1.3793$ g/mol

eq 1.40, $M = M_{\odot}$ $\implies$

$C = 6.65 \cdot 10^4 \frac{\mu}{Z(1+X)} = 91.7241 \cdot 10^4$ erg $s^{-1} K^{\frac{-7}{2}}$ 

$T_c = (\frac{L}{C})^{\frac{2}{7}}$

$L = 0.03 L_{\odot} = 0.117 \cdot 10^{33} $ erg/s $\implies$

$T_c = 2.0697 \cdot 10^6$ K

$T_s = (\frac{C}{4 \pi R^2 \sigma})^{\frac{1}{4}} T_c^{\frac{7}{8}}$

$\sigma = 5.67 \cdot 10^{-5}$ erg $cm^{-2} K^{-4} s^{-1}$

$R = R_{\odot} = 6.96 \cdot 10^{10}$ cm

$\implies T_s = 2412.9238$ K


\paragraph{H2 p4}

caso no relativista (baja densidad: $\rho << 6  \cdot 10^{15} $ g/$cm^3$)

$\gamma = 5/3$, $K = \frac{3^{\frac{2}{3}} \pi^{\frac{4}{3}} \hbar^2 }{5 m_n^{\frac{8}{3}}} = 5.38752 \cdot 10^9$

en la ecuación Lane Emden n = 1.5 igual que en el caso de las enanas blancas  de baja densidad $\implies $ tiene la misma resolución:
$\xi_1= 3.65375$ y $|\theta'(\xi_1)|= 0.203302$  

polítropos apuntes eq 1.18, 1.19:

$R = $


\paragraph{H3 p2}

partícula de masa = 1 parte del  reposo $\implies E = c^2$  (la energía total es la energía de su masa en reposo)

eq 3.6 apuntes $\implies (1-\frac{r_s}{r}) \frac{dt}{d\tau} = 1 \implies \frac{d\tau}{dt} = 1-\frac{r_s}{r}  $

apuntes (parte de una distancia R): $\tau(r) = \frac{1}{c} (\frac{R^3}{r_s})^{\frac{1}{2}} [(\frac{r}{R} -\frac{r^2}{R^2})^{\frac{1}{2}} + arccos(\sqrt{\frac{r}{R}}) ] $

$r = R \frac{1+cos \eta}{2} \implies \tau(\eta) = \frac{1}{c} (\frac{R^3}{r_s})^{\frac{1}{2}} [(\frac{1+cos \eta}{2} -(\frac{1+cos \eta}{2}) ^2)^{\frac{1}{2}} + arccos(\sqrt{\frac{1+cos \eta}{2}}) ] $

$\frac{d\tau}{d\eta} =  \frac{1}{c} (\frac{R^3}{r_s})^{\frac{1}{2}} ( \frac{\frac{1}{2} \sin (\eta) (\cos (\eta)+1)-\frac{\sin (\eta)}{2}}{2 \sqrt{\frac{1}{2} (\cos (\eta)+1)-\frac{1}{4} (\cos (\eta)+1)^2}} + \frac{\sin (\eta)}{2 \sqrt{2} \sqrt{\frac{1}{2} (-\cos (\eta)-1)+1} \sqrt{\cos (\eta)+1}} )$

$\frac{d\tau}{d\eta} \frac{d\eta}{dt} = 1-\frac{r_s}{r} \implies \frac{dt}{d\eta} =  \frac{1}{c} (\frac{R^3}{r_s})^{\frac{1}{2}} (1-\frac{2 r_s}{R(1+cos \eta)})^{-1} ( \frac{\frac{1}{2} \sin (\eta) (\cos (\eta)+1)-\frac{\sin (\eta)}{2}}{2 \sqrt{\frac{1}{2} (\cos (\eta)+1)-\frac{1}{4} (\cos (\eta)+1)^2}} + \frac{\sin (\eta)}{2 \sqrt{2} \sqrt{\frac{1}{2} (-\cos (\eta)-1)+1} \sqrt{\cos (\eta)+1}} ) $

$\implies t(\eta) = \frac{(\cos (\eta)+1)^{3/2} \tan \left(\frac{\eta}{2}\right) \sec ^2\left(\frac{\eta}{2}\right) \left(4 r_s^{3/2} \tanh ^{-1}\left(\frac{\sqrt{r_s} \tan \left(\frac{\eta}{2}\right)}{\sqrt{R-r_s}}\right)+\sqrt{R-r_s} (\eta (R+2 r_s)+R \sin (\eta))\right)}{4 R \sqrt{R-r_s} \sqrt{1-\cos (\eta)}}$

singularidad en $r=r_s$

\paragraph{H3 p4}

partícula con masa m = 1 parte del reposo ($E = c^2$) desde el infinito:

$\frac{dr}{d\tau} = -c (\frac{r_s}{r})^{\frac{1}{2}} \implies$

$r^{\frac{1}{2}} dr = -c r_s^{\frac{1}{2}} d\tau \implies$

$\tau(r) = C -\frac{2}{3}c^{-1}r_s^{-\frac{1}{2}} r^{\frac{3}{2}} $

$\tau(R) = 0 \implies C =  \frac{2}{3}c^{-1}r_s^{-\frac{1}{2}} R^{\frac{3}{2}} $

$\tau(r) = \frac{2}{3}c^{-1}r_s^{-\frac{1}{2}} R^{\frac{3}{2}}  -\frac{2}{3}c^{-1}r_s^{-\frac{1}{2}} r^{\frac{3}{2}} $

$\tau(r_s) = \frac{2}{3}c^{-1}r_s^{-\frac{1}{2}} R^{\frac{3}{2}}  -\frac{2}{3}c^{-1}r_s^{-\frac{1}{2}} r_s^{\frac{3}{2}} = \frac{2}{3}c^{-1}r_s^{-\frac{1}{2}} R^{\frac{3}{2}}  -\frac{2}{3}c^{-1}r_s$

\end{document}
