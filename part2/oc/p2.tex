\documentclass[10pt]{book}
\usepackage{graphicx}
\usepackage{subfig} % make it possible to include more than one captioned figure/table in a single float
\usepackage[utf8]{inputenc}
\usepackage{hyperref}
\usepackage[intlimits]{amsmath}
\usepackage{amssymb}
\usepackage{float}
\setlength{\oddsidemargin}{15.5pt} 
\setlength{\evensidemargin}{15.5pt}
\pretolerance=2000
\tolerance=3000
\renewcommand{\figurename}{Figura}
\renewcommand{\chaptername}{Cap\'{i}tulo}
\renewcommand{\contentsname}{\'{I}ndice}
\renewcommand{\tablename}{Tabla}
\renewcommand{\bibname}{Bibliograf\'{i}a}
\renewcommand{\appendixname}{Ap\'endices}


\usepackage{geometry}
 \geometry{
 a4paper,
 left=5mm,
 right=5mm,
 top=5mm,
 bottom=5mm,
 }

\begin{document}

\paragraph{H3 p2}

partícula de masa = 1 en reposo $\implies E = c^2$  (la energía total es la energía de su masa en reposo)

eq 3.6 apuntes $\implies (1-\frac{r_s}{r}) \frac{dt}{d\tau} = 1 \implies \frac{d\tau}{dt} = 1-\frac{r_s}{r}  $

apuntes: $\tau(r) = \frac{1}{c} (\frac{R^3}{r_s})^{\frac{1}{2}} [(\frac{r}{R} -\frac{r^2}{R^2})^{\frac{1}{2}} + arccos(\sqrt{\frac{r}{R}}) ] $

$r = R \frac{1+cos \eta}{2} \implies \tau(\eta) = \frac{1}{c} (\frac{R^3}{r_s})^{\frac{1}{2}} [(\frac{1+cos \eta}{2} -(\frac{1+cos \eta}{2}) ^2)^{\frac{1}{2}} + arccos(\sqrt{\frac{1+cos \eta}{2}}) ] $

$\frac{d\tau}{d\eta} =  \frac{1}{c} (\frac{R^3}{r_s})^{\frac{1}{2}} ( \frac{\frac{1}{2} \sin (x) (\cos (x)+1)-\frac{\sin (x)}{2}}{2 \sqrt{\frac{1}{2} (\cos (x)+1)-\frac{1}{4} (\cos (x)+1)^2}} + \frac{\sin (x)}{2 \sqrt{2} \sqrt{\frac{1}{2} (-\cos (x)-1)+1} \sqrt{\cos (x)+1}} )$

$\frac{d\tau}{d\eta} \frac{d\eta}{dt} = 1-\frac{r_s}{r} \implies \frac{dt}{d\eta} =  \frac{1}{c} (\frac{R^3}{r_s})^{\frac{1}{2}} (1-\frac{2 r_s}{R(1+cos \eta)})^{-1} ( \frac{\frac{1}{2} \sin (x) (\cos (x)+1)-\frac{\sin (x)}{2}}{2 \sqrt{\frac{1}{2} (\cos (x)+1)-\frac{1}{4} (\cos (x)+1)^2}} + \frac{\sin (x)}{2 \sqrt{2} \sqrt{\frac{1}{2} (-\cos (x)-1)+1} \sqrt{\cos (x)+1}} ) $

$\implies \tau(\eta) = \frac{(\cos (x)+1)^{3/2} \tan \left(\frac{x}{2}\right) \sec ^2\left(\frac{x}{2}\right) \left(4 \text{rs}^{3/2} \tanh ^{-1}\left(\frac{\sqrt{\text{rs}} \tan \left(\frac{x}{2}\right)}{\sqrt{R-\text{rs}}}\right)+\sqrt{R-\text{rs}} (x (R+2 \text{rs})+R \sin (x))\right)}{4 R \sqrt{R-\text{rs}} \sqrt{1-\cos (x)}}$

\end{document}
