\documentclass[tikz,border=1cm]{book} 
\usepackage[intlimits]{amsmath}
\usepackage{amssymb}
\usepackage{tikz}
\usepackage{tikz-3dplot} 
\begin{document}


\begin{equation}
	\vec{v_f} = \frac{1}{q} \frac{\vec{F} \times \vec{B}}{B^2} 
\end{equation}


\tdplotsetmaincoords{60}{150} 
\begin{tikzpicture} [scale=2, tdplot_main_coords, axis/.style={->,blue,thick}, 
vector/.style={-stealth,red,very thick}, 
vector guide/.style={dashed,red,thick}]

%standard tikz coordinate definition using x, y, z coords
\coordinate (O) at (0,0,0);

%tikz-3dplot coordinate definition using x, y, z coords


\coordinate (P) at (1,0,0);
\node[tdplot_main_coords,anchor=west] at (1,0,0){$\vec{B}$};
\coordinate (G) at (0,0,-1);
\node[tdplot_main_coords,anchor=east] at (0,0,-1){$\vec{g}$};
\node[tdplot_main_coords,anchor=south] at (0,0,0){O};
\coordinate (E) at (0,1,0);
\node[tdplot_main_coords,anchor=south] at (0,1,0){$\vec{u_e}$};
\coordinate (I) at (0,-1,0);
\node[tdplot_main_coords,anchor=south] at (0,-1,0){$\vec{u_i}$};

%draw axes
\draw[axis] (0,0,0) -- (2,0,0) node[anchor=north east]{$x$};
\draw[axis] (0,0,0) -- (0,2,0) node[anchor=north west]{$y$};
\draw[axis] (0,0,0) -- (0,0,2) node[anchor=south]{$z$};

%draw a vector from O to P
\draw[vector] (O) -- (P);
\draw[vector] (O) -- (G);
\draw[vector] (O) -- (E);
\draw[vector] (O) -- (I);

\end{tikzpicture}


\tdplotsetmaincoords{60}{150} 
\begin{tikzpicture} [scale=2, tdplot_main_coords, axis/.style={->,blue,thick}, 
vector/.style={-stealth,red,very thick}, 
vector guide/.style={dashed,red,thick}]

%standard tikz coordinate definition using x, y, z coords
\coordinate (O) at (0,0,0);

%tikz-3dplot coordinate definition using x, y, z coords


\coordinate (P) at (1,0,0);
\node[tdplot_main_coords,anchor=west] at (1,0,0){$\vec{B}$};
\coordinate (G) at (0,0,-1);
\node[tdplot_main_coords,anchor=east] at (0,0,-1){$\vec{g}$};
\coordinate (L) at (0,0,1);
\node[tdplot_main_coords,anchor=east] at (0,0,1){$\vec{F_L} = \vec{J} \times \vec{B}$};
\node[tdplot_main_coords,anchor=south] at (0,0,0){O};
\coordinate (I) at (0,-1,0);
\node[tdplot_main_coords,anchor=south] at (0,-1,0){$\vec{J}$};

%draw axes
\draw[axis] (0,0,0) -- (2,0,0) node[anchor=north east]{$x$};
\draw[axis] (0,0,0) -- (0,2,0) node[anchor=north west]{$y$};
\draw[axis] (0,0,0) -- (0,0,2) node[anchor=south]{$z$};

%draw a vector from O to P
\draw[vector] (O) -- (P);
\draw[vector] (O) -- (G);
\draw[vector] (O) -- (L);
\draw[vector] (O) -- (I);

\end{tikzpicture}

5 species: electrons, H , $H^+$,  He, $He^+$

temperature and density constant?

Horizontal drifts(for charges) increased by frictional forces which in this case sum to gravity
(neutrals fall more quickly - vertical frictional force upward and for charges opposite) 

Vertical drifts due to horizontal collisional forces:

Downwards
ions ($He^+$):

\tdplotsetmaincoords{60}{150} 
\begin{tikzpicture} [scale=2, tdplot_main_coords, axis/.style={->,blue,thick}, 
vector/.style={-stealth,red,very thick}, 
vector guide/.style={dashed,red,thick}]

%standard tikz coordinate definition using x, y, z coords
\coordinate (O) at (0,0,0);

%tikz-3dplot coordinate definition using x, y, z coords


\coordinate (P) at (1,0,0);
\node[tdplot_main_coords,anchor=west] at (1,0,0){$\vec{B}$};
\node[tdplot_main_coords,anchor=south] at (0,0,0){O};

\coordinate (F) at (0,1,0);
\node[tdplot_main_coords,anchor=south] at (0,1,0){$\vec{F_c}$};

\coordinate (D) at (0,0,-1);
\node[tdplot_main_coords,anchor=south] at (0,0,-1){$\vec{u_d}$};

%draw axes
\draw[axis] (0,0,0) -- (2,0,0) node[anchor=north east]{$x$};
\draw[axis] (0,0,0) -- (0,2,0) node[anchor=north west]{$y$};
\draw[axis] (0,0,0) -- (0,0,2) node[anchor=south]{$z$};

%draw a vector from O to P
\draw[vector] (O) -- (P);
\draw[vector] (O) -- (F);
\draw[vector] (O) -- (D);

\end{tikzpicture}

electrons:

\tdplotsetmaincoords{60}{150} 
\begin{tikzpicture} [scale=2, tdplot_main_coords, axis/.style={->,blue,thick}, 
vector/.style={-stealth,red,very thick}, 
vector guide/.style={dashed,red,thick}]

%standard tikz coordinate definition using x, y, z coords
\coordinate (O) at (0,0,0);

%tikz-3dplot coordinate definition using x, y, z coords


\coordinate (P) at (1,0,0);
\node[tdplot_main_coords,anchor=west] at (1,0,0){$\vec{B}$};
\node[tdplot_main_coords,anchor=south] at (0,0,0){O};

\coordinate (F) at (0,-1,0);
\node[tdplot_main_coords,anchor=south] at (0,-1,0){$\vec{F_c}$};

\coordinate (D) at (0,0,-1);
\node[tdplot_main_coords,anchor=south] at (0,0,-1){$\vec{u_d}$};

%draw axes
\draw[axis] (0,0,0) -- (2,0,0) node[anchor=north east]{$x$};
\draw[axis] (0,0,0) -- (0,2,0) node[anchor=north west]{$y$};
\draw[axis] (0,0,0) -- (0,0,2) node[anchor=south]{$z$};

%draw a vector from O to P
\draw[vector] (O) -- (P);
\draw[vector] (O) -- (F);
\draw[vector] (O) -- (D);

\end{tikzpicture}

Upwards : ions  ($H^{+}$)

\tdplotsetmaincoords{60}{150} 
\begin{tikzpicture} [scale=2, tdplot_main_coords, axis/.style={->,blue,thick}, 
vector/.style={-stealth,red,very thick}, 
vector guide/.style={dashed,red,thick}]

%standard tikz coordinate definition using x, y, z coords
\coordinate (O) at (0,0,0);

%tikz-3dplot coordinate definition using x, y, z coords


\coordinate (P) at (1,0,0);
\node[tdplot_main_coords,anchor=west] at (1,0,0){$\vec{B}$};
\node[tdplot_main_coords,anchor=south] at (0,0,0){O};

\coordinate (F) at (0,-1,0);
\node[tdplot_main_coords,anchor=south] at (0,-1,0){$\vec{F_c}$};

\coordinate (D) at (0,0,1);
\node[tdplot_main_coords,anchor=south] at (0,0,1){$\vec{u_d}$};

%draw axes
\draw[axis] (0,0,0) -- (2,0,0) node[anchor=north east]{$x$};
\draw[axis] (0,0,0) -- (0,2,0) node[anchor=north west]{$y$};
\draw[axis] (0,0,0) -- (0,0,2) node[anchor=south]{$z$};

%draw a vector from O to P
\draw[vector] (O) -- (P);
\draw[vector] (O) -- (F);
\draw[vector] (O) -- (D);

\end{tikzpicture}

\end{document}
