\documentclass[10pt]{book}
\usepackage{graphicx}
\usepackage{subfig} % make it possible to include more than one captioned figure/table in a single float
\usepackage[utf8]{inputenc}
\usepackage{hyperref}
\usepackage[intlimits]{amsmath}
\usepackage{amssymb}
\setlength{\oddsidemargin}{15.5pt} 
\setlength{\evensidemargin}{15.5pt}
\pretolerance=2000
\tolerance=3000
\renewcommand{\figurename}{Figura}
\renewcommand{\chaptername}{Cap\'{i}tulo}
\renewcommand{\contentsname}{\'{I}ndice}
\renewcommand{\tablename}{Tabla}
\renewcommand{\bibname}{Bibliograf\'{i}a}
\renewcommand{\appendixname}{Ap\'endices}


\usepackage{geometry}
 \geometry{
 a4paper,
 left=5mm,
 right=5mm,
 top=5mm,
 bottom=5mm,
 }

\begin{document}

\paragraph {1a)} 
\begin{figure}[!ht]
 \centering
 \includegraphics[scale=0.5]{tempLayers.png}
 \caption{\emph{Temperature vs z plot. logarithmic y scale}}
\end{figure}

In order to identify the layers I put conditions on temperature:

\url{http://www.nasa.gov/mission_pages/iris/multimedia/layerzoo.html}

Looking at  the values from the  file 'atmosphere.dat' ordered by height from top(of the atmosphere) 
to bottom I consider the corona while temperature $\ge 500000$ K (T is decreasing), 
transition region until T = 8000 K, the chromosphere until T reaches the (only) minimum, afterwards the temperature starts to raise
and I consider the layer before it reaches 6500 K the photosphere and  the solar interior after

The exact values matching these conditions are: 
\begin{description}

\item corona between [39.802200, 2.535930] Mm temperatures: [1.080180e+06, 5.025160e+05] K 
\item transition region between [2.516350, 0.991115] Mm temperatures: [4.991350e+05, 8.067640e+03] K  
\item chromosphere between [0.971556, 0.305708] Mm temperatures: [7.306160e+03, 2.843670e+03] K 
\item photosphere between [0.286093, -0.303487] Mm temperatures: [2.848470e+03, 6.297540e+03] K 
\item solar interior between [-0.323184, -2.592960] Mm temperatures: [6.837750e+03, 2.068340e+04] K 

\end{description}



\paragraph {1b)}

$\mu = \frac{n_{H} + 4 n_{He}}{n_e + n_H + n_{He}} $

$n_H= 10 n_{He} \implies  \mu = \frac{1.4 n_H}{n_e + 1.1 n_H}$


\begin{itemize}
\item totally ionized H and He $\implies n_e = n_H + 2 n_{He} = 1.2 n_H \implies \frac{n_H}{n_e} = \frac{5}{6} $ and 
$ \mu = 0.6987 $
\item neutral H and He $\implies n_e = 0 \implies  \mu = 1.2727 $

\end{itemize}



\begin{figure}[!ht]
 \centering
 \includegraphics[scale=0.5]{mmmLayers.png}
 \caption{\emph{Mean molecular weight(g/mol) vs z plot} Maximum close to 1.2727 = $\mu$ in the case of neutral H and He and minimum close to 0.6087 = $\mu$ calculated in the case of completely ionized H and He }
\end{figure}

$\frac{n_H}{n_e} =  \frac{\mu}{1.4 - 1.1 \mu }   $

In the case of neutral H and He $n_e \rightarrow 0 \implies \frac{n_H}{n_e} \rightarrow \infty $

When plotting  $\frac{n_H}{n_e}$ using $\mu$ from the file, as we can see in the graphic of $\mu$ there are some values of z for which
$\mu > 1.2727 \implies 1.4 - 1.1 \mu < 0 \implies \frac{n_H}{n_e} < 0$

I will limit oy axis values to $ [0,4]$ 


\begin{figure}[!ht]
 \centering
 \includegraphics[scale=0.5]{nHDivNe.png}
 \caption{number of atoms of H / number of electrons }
\end{figure}

We can see a constant value of $\frac{n_H}{n_e}$ in the corona of $\frac{n_H}{n_e} = 0.843 \approx  \frac{5}{6}$ 
which is the value we calculate in the case of totally ionized H and He and we expect this because of the high values of
the temperature in the corona



\begin{figure}[!ht]
 \centering
 \includegraphics[scale=0.5]{hpLayers.png}
 \caption{Pressure scale height}
\end{figure}


\paragraph{2)}

$\frac{d \text{ln} p}{dz} = -\frac{1}{H_p}$, $H_p$ const $\implies$ ln $p(z)$ - ln $ p(z_0) = -\frac{1}{H_p}(z-z_0) \implies p(z) = p(z_0) exp(-\frac{z-z_0}{H_p})$  

$\rho(z) =  \frac{1}{g H_p}p(z) = \frac{p(z_0)}{g H_p} exp(-\frac{z-z_0}{H_p}) = \rho(z_0) exp(-\frac{z-z_0}{H_p})$ 

Analytic test for $H_p$ constant (with values 1 and $1e10$) with $\rho(z_{max})$ taking values: $1e-10, 1e-5, 1e-2,1, 1e2,1e3, 1e7, 1e10$
Integrating downward or forward in height makes no difference (using ln p)
\newpage 

\begin{figure}[!ht]
 \centering
 \includegraphics[scale=0.5]{allanalytic.png}
 \caption{Analytic test Hp=1}
\end{figure}

\newpage 

\begin{figure}[!ht]
 \centering
 \includegraphics[scale=0.5]{allanalytic2.png}
 \caption{Analytic test Hp=1e10}
\end{figure}

\newpage

\begin{figure}[!ht]
 \centering
 \includegraphics[scale=0.5]{fromFileLn1.png}
 \caption{logarithmic (base e: ln) of pres, rho}
\end{figure}

\begin{figure}[!ht]
 \centering
 \includegraphics[scale=0.5]{fromFile1.png}
 \caption{pres, rho}
\end{figure}

Notation: $\mu_0$ = magnetic permeability 

$\beta = \frac{p}{p_{mag}}$ where $p_{mag} = \frac{B^2}{2 \mu_0}$

$v_A = \frac{B ^2 }{ \mu_0 \rho} $

$c_s = \sqrt{\frac{\gamma p}{ \rho}} $

\newpage
\begin{figure}[!ht]
 \centering
 \includegraphics[scale=0.5]{fromFileLn2.png}
 \caption{logarithmic (base e: ln) of beta plasma, vA, cs}
\end{figure}

\begin{figure}[!ht]
 \centering
 \includegraphics[scale=0.5]{fromFile2.png}
 \caption{beta plasma, vA, cs}
\end{figure}

$\beta = \frac{2 p \mu_0}{B^2} = \frac{2 p}{\rho v_A^2} = \frac{2 }{\gamma}(\frac{c_s}{v_A})^2
\implies \beta (\frac{v_A}{c_s})^2 \frac{\gamma}{2} = 1 $
We call this function func($\beta, \frac{v_A}{c_s}$) in the graphic below and expect it to be 1


\newpage

\begin{figure}[!ht]
 \centering
 \includegraphics[scale=0.5]{fromFileLn3.png}
 \caption{ln func(bp, vA/cs) $\approx$ 0}
\end{figure}


\paragraph{3a)}

\begin{figure}[!ht]
 \centering
 \includegraphics[scale=0.5]{lambdaPh.png}
 \caption{ Lambda phot}
\end{figure}

\begin{figure}[!ht]
 \centering
 \includegraphics[scale=0.5]{lambdaC.png}
 \caption{ Lambda corona}
\end{figure}

Both functions have the maximum for T = 2.238721e+05 K


\begin{figure}[!ht]
 \centering
 \includegraphics[scale=0.5]{interpLambdaC.png}
 \caption{ Lambda corona interpolated for atm. temperatures $> 3 * 10^4$ K in 'atmosphere.dat' plotted vs z}
\end{figure}

\begin{figure}[!ht]
 \centering
 \includegraphics[scale=0.5]{Lr.png}
 \caption{ Lr logarithmic y scale}
\end{figure}

\begin{figure}[!ht]
 \centering
 \includegraphics[scale=0.5]{ue.png}
 \caption{ Internal energy}
\end{figure}

\begin{figure}[!ht]
 \centering
 \includegraphics[scale=0.5]{ueDivLr.png}
 \caption{ Internal energy / Lr}
\end{figure}
\end{document}
